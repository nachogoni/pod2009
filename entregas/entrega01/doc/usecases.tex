% Defino el tipo de documento.
\documentclass[a4paper,11pt]{article}

% Este paquete permite hacer encabezados y pies de página como
% Los de las guias
%\usepackage{fancyhdr}

\usepackage[spanish]{babel}
\usepackage[utf8]{inputenc}

% Este es para poder poner graficos y diagramas.
\usepackage{graphicx}

% Este paquete está por las dudas... Si algun archivo eps se vuelve
% rebelde y no sale donde uno quiere, hay que encajarle "\FloatBarrier" antes
% y despues y listo.
\usepackage{placeins}

% Este paquete me permite poner grados celsius.
\usepackage{textcomp}


%====================
%= TEMPLATES UTILES =
%====================

%\begin{table}[htpb]
%	\centering
%	\begin{tabular}{|c|c|} \hline
%	Cant. Iter. M\'aximas & Tiempo(seg) \\ \hline
%	100			&	13.80 \\
%	1000		&	14.56 \\
%	2500		&	16.16 \\
%	5000		&	18.48 \\
%	10000		&	23.05 \\
%	25000		&	37.21 \\
%	50000		&	60.43 \\
%	100000		&	106.99 \\ \hline
%	\end{tabular}
%	\caption{\label{ej02Tiempos}Tiempos medido para distintas iteraciones máximas del conjunto de Mandelbrot.}
%\end{table}

%\begin{figure}[htpb]
%\centering
%\includegraphics[width=9cm, height=9cm]{ej03local.png}
%\caption{\label{ej03local} Curva de tiempos en segundos para los primeros 12 casos  }
%\end{figure}

% Toda la configuración del documento.
%\input{preambulo.tex}

%Encabeados y pie de pagina
%\pagestyle{headings}

\title{
        Programación de Objetos Distribuidos \\
        Entrega \#1: \\
        Diagrama y Casos de Uso
    }

\author{
        Grupo 3. \\
        46099 - Abramowicz, Pablo Federico \\
        46281 - Cabral, Martín Esteban \\
        47031 - Gomez, Vidal Darío Maximiliano \\
        46383 - Goñi, Juan Ignacio \\
        46233 - Palombo, Martín \\
        46069 - Sessa, Carlos Manuel
        }
\date{}

% Empieza el documento.
\begin{document}


\maketitle
\pagebreak
\section{How to write use cases}

Definiciones útiles para completar el template
(alguna aplican, otras no, tomen lo que necesiten)

\begin{itemize}
 \item Actor: Cualquiera o cualquier cosa con comportamiento.
 \item Stakeholder: Alguien o algo con interés en el comportamiento del sistema.
 \item Primary Actor: El Stakeholder que inicia una interacción con el sistema para alcanzar una meta.
 \item Use case: un contrato de comportamiento del sistema.
 \item Scope: identifica al sistema que estamos discutiendo.
 \item Precondiciones: Cosas que deben ser true antes del que caso de uso corra.
 \item Garantías: Post condiciones.
 \item Main Success Scenario: El caso feliz.
 \item Extensions: Que puede pasar diferente en el caso feliz.
\end{itemize}

Los caso de uso que son 'incluidos' aparecen subrayados.

Ej. '2.3.1 El usuario hace click en el boton Save para \underline{Guardar el trabajo}'.

\pagebreak
\section{Template}
Usamos el \emph{RUP Style}. Tratemos de poner nombres de caso de uso 'descriptivos'.
Fíjense que para que si necesitan agregar un nivel de anidación tiene que agregar
las líneas que se pueden ver en el .tex

%Redefine the first level
\renewcommand{\theenumi}{\arabic{enumi}.}
\renewcommand{\labelenumi}{\theenumi}
 
%Redefine the second level
\renewcommand{\theenumii}{\arabic{enumii}.}
\renewcommand{\labelenumii}{\theenumii}

%Redefine the thrid level
\renewcommand{\theenumiii}{\arabic{enumiii}.}
\renewcommand{\labelenumiii}{\theenumiii}

\begin{enumerate}

    \item Nombre del caso de uso: \\
    Ver reservas de canchas.

    \begin{enumerate}
    \item Descripción breve: \\
        Se listan las reservas de canchas.
    \item Actores \\
        Administrador.
    \item Disparadores: \\
        Click en el botón \underline{Ver reservas de canchas} dentro de la
        página principal del administrador.
    \end{enumerate}

    \item Flujo de Eventos: \\

    \begin{enumerate}

        \item Flujo básico:\\
            Un usuario genérico hace \underline{Login} como administrador y
            hace click en \underline{Ver reservas de canchas}.
        \item Flujos alternativos:\\
            Se muestra un cartel que no existen reservas de cancha.
            \begin{enumerate}
                \item Condición 1 \\
                    No existan reservas de canchas.
            \end{enumerate}

    \end{enumerate}

    \item Precondiciones: \\
        Administrador logueado como administrador.

    \item Post condiciones \\
        No aplica.

\end{enumerate}

% 46099 - Abramowicz, Pablo Federico \\ INICIO
% 46099 - Abramowicz, Pablo Federico \\ FIN

% 46281 - Cabral, Martín Esteban \\ INICIO
% 46281 - Cabral, Martín Esteban \\ FIN

% 47031 - Gomez, Vidal Darío Maximiliano \\ INICIO
% 47031 - Gomez, Vidal Darío Maximiliano \\ FIN

% 46383 - Goñi, Juan Ignacio \\ INICIO

% reservas caidas rss

% nuevas canchas rss

% invitado registrarse

% invitado login

% 46383 - Goñi, Juan Ignacio \\ FIN

% 46233 - Palombo, Martín \\ INICIO
\begin{enumerate}

    \item Nombre del caso de uso: \\
    Cargar pago reservas.

    \begin{enumerate}
    \item Descripción breve: \\
        Se carga un pago para una determinada reserva.
    \item Actores \\
        Administrador.
    \item Disparadores: \\
        Click en el botón \underline{Cargar pago de reserva} dentro de la
        página que muestra las reservas para una cancha.
    \end{enumerate}

    \item Flujo de Eventos: \\

    \begin{enumerate}

        \item Flujo básico:\\
            Un usuario genérico hace \underline{Login} como administrador y
            hace click en \underline{Ver reservas de canchas}. Luego para una
            reserva particular hace click en
            \underline{Cargar pago de reserva}. El administrador ingresa el
            monto que fue pagado de la reserva. En el caso de que se logre el
            mínimo para pasar a estado señada se pasa la reserva a dicho
            estado, lo mismo si se completa el pago total, pasándola a estado
            pagada; en ambos casos se actualizan los puntos del usuario que
            emitió la reserva y se informa al usuario que la operación fue
            exitosa. En caso contrario, ya que la cantidad ingresada es
            incorrecta, se procede al \underline{Flujo alternativo 1}. 


        \item Flujos alternativos:\\
            Se muestra un cartel que informa que la cantidad ingresada es
            incorrecta.
            \begin{enumerate}
                \item Condición 1 \\
                    La cantidad ingresada es incorrecta.
            \end{enumerate}

    \end{enumerate}

    \item Precondiciones: \\
        Ususario logueado como administrador.

    \item Post condiciones \\
        Pago de la reserva cargado y, si aplica, estado de la reserva y puntos
        del usuario que emitió la reserva actualizados.

\end{enumerate}

\begin{enumerate}

    \item Nombre del caso de uso: \\
    Cargar reserva completa.

    \begin{enumerate}
    \item Descripción breve: \\
        Se carga un pago completo para una determinada reserva.
    \item Actores \\
        Administrador.
    \item Disparadores: \\
        Click en el botón \underline{Cargar pago completo de reserva} dentro de la
        página que muestra las reservas para una cancha.
    \end{enumerate}

    \item Flujo de Eventos: \\

    \begin{enumerate}

        \item Flujo básico:\\
            Un usuario genérico hace \underline{Login} como administrador y
            hace click en \underline{Ver reservas de canchas}. Luego para una
            reserva particular hace click en
            \underline{Cargar pago completo de reserva}. La reserva pasa al
            estado pagada y se actualizan los puntos del usuario que emitió la
            reserva. Se informa al usuario que la operación fue exitosa.

    \end{enumerate}

    \item Precondiciones: \\
        Ususario logueado como administrador.

    \item Post condiciones \\
        Pago de la reserva cargado y estado de la reserva actualizado a pagada.
        Puntos del usuario que emitió la reserva actualizados.

\end{enumerate}

\begin{enumerate}

    \item Nombre del caso de uso: \\
    Cancelar reserva.

    \begin{enumerate}
    \item Descripción breve: \\
        Se da de baja una determinada reserva.
    \item Actores \\
        Usuario logueado como usuario común.
    \item Disparadores: \\
        Click en el botón \underline{Cancelar reserva} dentro de la
        página que muestra las reservas de una cancha.
    \end{enumerate}

    \item Flujo de Eventos: \\

    \begin{enumerate}

        \item Flujo básico:\\
            Un usuario genérico hace \underline{Login} como usuario común y
            hace click en \underline{Ver reservas de canchas}. Luego para una
            reserva particular hace click en \underline{Cancelar reserva}.
            La reserva se elimina del sistema. Se informa al usuario que la
            operación fue exitosa.

    \end{enumerate}

    \item Precondiciones: \\
        Ususario logueado como usuario común.

    \item Post condiciones \\
        Reserva dada de baja.

\end{enumerate}

\begin{enumerate}

    \item Nombre del caso de uso: \\
    Ver detalle complejo.

    \begin{enumerate}
    \item Descripción breve: \\
        Se muestra en detalle las características de un complejo.
    \item Actores \\
        Administrador.
    \item Disparadores: \\
        Click en el botón \underline{Ver detalle} dentro de la
        página que lista los complejos del sistema.
    \end{enumerate}

    \item Flujo de Eventos: \\

    \begin{enumerate}

        \item Flujo básico:\\
            Un usuario genérico hace \underline{Login} como usuario común y
            hace click en \underline{Listar complejos}. Se visualiza el
            complejo seleccionado, las canchas que tiene, la información del
            mismo y todos los datos que contiene. A su vez tiene el detalle de
            cada cancha, el cual consiste de sus características y los horarios
            disponibles en un calendario.


    \end{enumerate}

    \item Precondiciones: \\
        Usuario logueado como usuario común.

    \item Post condiciones \\
        No aplica.

\end{enumerate}
% 46233 - Palombo, Martín \\ FIN

% 46069 - Sessa, Carlos Manuel \\ INICIO
% 46069 - Sessa, Carlos Manuel \\ FIN

\end{document}
