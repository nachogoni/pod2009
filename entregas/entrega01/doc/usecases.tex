% Defino el tipo de documento.
\documentclass[a4paper,11pt]{article}

% Este paquete permite hacer encabezados y pies de p�gina como
% Los de las guias
%\usepackage{fancyhdr}

% Es es para que sea sudaca-friendly
\usepackage[spanish]{babel}
\usepackage[latin1]{inputenc}

% Este es para poder poner graficos y diagramas.
\usepackage{graphicx}

% Este paquete est� por las dudas... Si algun archivo eps se vuelve
% rebelde y no sale donde uno quiere, hay que encajarle "\FloatBarrier" antes
% y despues y listo.
\usepackage{placeins}

% Este paquete me permite poner grados celsius.
\usepackage{textcomp}


%====================
%= TEMPLATES UTILES =
%====================

%\begin{table}[htpb]
%	\centering
%	\begin{tabular}{|c|c|} \hline
%	Cant. Iter. M\'aximas & Tiempo(seg) \\ \hline
%	100			&	13.80 \\
%	1000		&	14.56 \\
%	2500		&	16.16 \\
%	5000		&	18.48 \\
%	10000		&	23.05 \\
%	25000		&	37.21 \\
%	50000		&	60.43 \\
%	100000		&	106.99 \\ \hline
%	\end{tabular}
%	\caption{\label{ej02Tiempos}Tiempos medido para distintas iteraciones m\'aximas del conjunto de Mandelbrot.}
%\end{table}

%\begin{figure}[htpb]
%\centering
%\includegraphics[width=9cm, height=9cm]{ej03local.png}
%\caption{\label{ej03local} Curva de tiempos en segundos para los primeros 12 casos  }
%\end{figure}

% Toda la configuraci�n del documento.
%\input{preambulo.tex}

%Encabeados y pie de pagina
%\pagestyle{headings}

\title{ Programacion de Objetos Distribuidos \\ Diagrama y Casos de Uso }

\author {Grupo 3}

\date{}

% Empieza el documento.
\begin{document}
 

\maketitle
\pagebreak
\section{How to write big dicked use cases (Oppossed to Pablos 'no dick' way)}

Definiciones utiles para completar el template (alguna aplican, otras no, tomen lo que necesiten)

\begin{itemize}
 \item Actor: Cualquiera o cualquier cosa con comportamiento
 \item Stakeholder: Alguien o algo con interes en el comportamiento del sistema
 \item Primary Actor: El Stakeholder que inicia una interaccion con el sistema para alcanzar un gol
 \item Use case: un contrato de comportamiento del sistema
 \item Scope: identifica al sistema que estamos discutiendo
 \item Precondiciones: Cosas que deben ser true antes del que caso de uso corra
 \item Garantias: PostCondiciones
 \item Main Success Scenario: El caso feliz
 \item Extensions: Que puede pasar diferente en el caso feliz
\end{itemize}

Los caso de uso que son 'incluidos' aparecen subrayados.

Ej. '2.3.1 El usuario hace click en el boton Save para \underline{Guardar el trabajo}'

\pagebreak
\section{Template}
Usamos el \emph{RUP Style}. Tratemos de poner nombres de caso de uso 'descriptivos'.
Fijense que para que si necesitan agregar un nivel de anidacion tiene que agregar las lineas que se pueden ver en el .tex

%Redefine the first level
\renewcommand{\theenumi}{\arabic{enumi}.}
\renewcommand{\labelenumi}{\theenumi}
 
%Redefine the second level
\renewcommand{\theenumii}{\arabic{enumii}.}
\renewcommand{\labelenumii}{\theenumii}

%Redefine the thrid level
\renewcommand{\theenumiii}{\arabic{enumiii}.}
\renewcommand{\labelenumiii}{\theenumiii}


\begin{enumerate}

 \item Use Case Name \\
...text...

\begin{enumerate}

\item Brief Description \\
...text...
\item Actors \\
...text...
\item Triggers \\
...text...

\end{enumerate} 

\item Flow of Events \\

\begin{enumerate}

\item Basic Flow \\
...text...
\item Alternative Flows \\

\begin{enumerate}

\item Condition 1 \\
...text...
\item Condition 2 \\
...text...
\item Etc \\
...text...

\end{enumerate}
\end{enumerate} 

\item Preconditions \\
...text...

\item PostConditions \\
...text...

\end{enumerate}



\end{document}
