% Defino el tipo de documento.
\documentclass[a4paper,11pt]{article}
\usepackage[spanish]{babel}
\usepackage[utf8]{inputenc}
% Este es para poder poner graficos y diagramas.
\usepackage{graphicx}
% Este paquete me permite poner grados celsius.
\usepackage{textcomp}

\title{
        Programación de Objetos Distribuidos \\
        Entrega \#3: \\
        Implementación de Casos de Uso
    }

\author{
        Grupo 3. \\
        46099 - Abramowicz, Pablo Federico \\
        46281 - Cabral, Martín Esteban \\
        47031 - Gomez, Vidal Darío Maximiliano \\
        46383 - Goñi, Juan Ignacio \\
        46233 - Palombo, Martín \\
        46069 - Sessa, Carlos Manuel
        }
\date{}

% Empieza el documento.
\begin{document}

\maketitle
\pagebreak

\section{Introducción}
En este documento se detallan decisiones tomadas para la implementación de los
casos de uso \emph{Administrar Canchas}, \emph{Administrar Complejos},
\emph{Hacer Reserva}, \emph{Listar Reservas}. Asimismo, se comentan agregados
que se realizaron para la mejor escalabilidad del proyecto y un más eficiente y
rápido desarrollo.

\section{URLMapper}
Al enfrentarnos con el problema de explicitar URLs en cada controlador
surgieron varias preguntas.
\begin{itemize}
 \item ¿Qué pasa si el día de mañana cambiamos el nombre o la ubicación
del controlador?
 \item ¿Qué sucede con la ubicación de los JSP?
 \item ¿Comó impactará el cambio de un archivo o el renombramiento del mismo?
 \item ¿Qué pasa si cambiamos la raíz de la aplicación?
\end{itemize}

No nos pareció correcto explicitar las rutas completas (o relativas desde
algún punto) en cada controlador para su respectivo \emph{redirect} o
\emph{forward}. Suponiendo que cambiamos la carpeta con los JSP de
lugar, teníamos que editar todos los \emph{paths} que usabamos antes.

Surgió la idea de crear un \emph{helper} que se encargue de hacer los
mapeos de recurso a URL por nosotros. Así surgió URLMapper.
Basta registrar una entrada indicando el URL destino y el método
(POST o GET) y ya podemos redireccionar o forwardear a la ubicación
deseada. Ante alguna modificación de ubicación el cambio sólo afectaría
a un archivo, minimizando efectos colaterales.

\section{FormHandler}
Cuando escribimos un formulario en JSP y el mismo quedo de unas aproximadas
mil lineas \emph{hardcodeadas}. Nos dimos cuenta que algo no estaba bien,
porque estabamos frente a algo inmantenible en el tiempo.
Se desarrollo para subsanar este problema un FormHandler, inspirado en su
totalidad en ZendForm, un elemento formulario que provee Zend framework
para PHP \footnote{http://framework.zend.com/manual/en/zend.form.html}.
Aprovechando que la totalidad de los integrantes lo habían usado,
implementamos la misma API acelerando el proceso de aprendizaje.

FormHandler es un contenedor de \emph{FormElement}.
Zend Form cuenta con más de 10 tipos de elementos.
Para esta entrega implementamos tres tipos de elementos:

\begin{itemize}
 \item Campo de Texto.
 \item Select.
 \item Botones de Submit.
\end{itemize}

Tanto el formulario en sí como sus FormElement cuentan con los helpers
de Zend Form. Entre ellos:
\begin{itemize}
 \item \emph{Validators} a las entradas.
\item \emph{Decorators} genéricos.
\item \emph{Display Groups} para imprimir los elementos en \emph{fieldsets}.
\end{itemize}

Con FormHandler nos aseguramos de no tener que escribir código
HTML reiteradamente.
Hizo que los JSP tengan un código mucho más mantenible.

\section{JQuery}
Para facilitar todo el manejo de JavaScript utilizamos la librería JQuery.
Implementaciones:
\begin{itemize}
 \item \emph{Calendars}.
\item \emph{Alerts}.
\end{itemize}

\section{Log4J}
No era requisito de esta entrega, pero decidimos tener Log4J funcionando
desde el primer día. Esto nos ayudó mucho a la hora de debuggear el código.

\end{document}
