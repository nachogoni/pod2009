% Defino el tipo de documento.
\documentclass[a4paper,11pt]{article}
\usepackage[spanish]{babel}
\usepackage[utf8]{inputenc}
% Este es para poder poner graficos y diagramas.
\usepackage{graphicx}
% Este paquete me permite poner grados celsius.
\usepackage{textcomp}

\title{
        Programación de Objetos Distribuidos \\
        Entrega \#4: \\
        Implementación de Casos de Uso
    }

\author{
        Grupo 3. \\
        46099 - Abramowicz, Pablo Federico \\
        46281 - Cabral, Martín Esteban \\
        47031 - Gomez Vidal, Darío Maximiliano \\
        46383 - Goñi, Juan Ignacio \\
        46233 - Palombo, Martín \\
        46069 - Sessa, Carlos Manuel
        }
\date{}

% Empieza el documento.
\begin{document}

\maketitle
\pagebreak

\section{Introducción}
En este documento se detallan algunas de las decisiones tomadas para la 
implementación de los casos de uso de la cuarta entrega.
Asimismo, se comentan agregados que se realizaron para la mejor
escalabilidad del proyecto y un más eficiente y rápido desarrollo.

\section{ConnectionManager}
Módulo que carga el driver de oracle y genera la conexión con la base de datos.
Mantiene una conexión abierta a la cual acceden los distintos threads para
mostrar resultados.

\section{ConnectionPool}
Al empezar a levantar información de la base de datos usando un único
\textit{ConnectionManager} notamos que el sitio funcionaba muy lento.
Para resolver este conflicto creamos un \textit{listener} que al iniciar el
servidor genere un pool de \textit{ConnectionManager}. Los distintos threads
piden \textit{ConnectionManager} al pool y cuando se cierra el servidor
libera todas las conexiones realizadas.

\section{ACL}
Realizamos un módulo de ACL, imponiendo roles y recursos.
De esta manera limitamos el acceso.

\section{FormHandler}
Desde la última entrega el FormHandler se actualizó para que entienda de
\textit{JavaScript}. Para realizar esto creamos la librería \emph{J2Query}
que se explicará a continuación.

\subsection{J2Query}
El siguiente módulo permite manejo de JS con JQuery dentro del
\textit{FormHandler}.

Mejoras ofrecidas por esta librería:
\subsubsection{Campo múltiple}
La idea de este campo es que se pueda agregar inputs en el formulario
a medida que el usuario lo requiera. Por ejemplo, cuando se agrega
un complejo se requiere ingresar más de un teléfono.

\subsubsection{Time Picker}
En un principio para la selección de fechas utilizamos combos pero notamos
que era molesto a la hora de una inserción masiva. Para solucionar esto,
decidimos que los campos de fechas sean texto, pero cuenten con un
widget de js para facilitar la inserción.

\subsubsection{Tooltips}
Notamos que alguno de los campos de los fórmularios no tenían labels
representativos, entonces utilizamos tooltips para facilitar el entedimiento
de los datos a cargar por el usuario.

\subsubsection{Futuras mejoras}
Por falta de tiempo no pudimos terminar todas las ideas que teníamos para
este módulo. Una de ellas, que creemos fundamental, es utilizar la información
de los validadores realizados para que se agregue verificación de los campos
del lado del cliente.

\section{Log4J}
Se decidió utilizar un SMTPAppender para enviar un correo electrónico a los
administradores del sitio ante un error fatal. Se mantuvo además la funcionalidad
presente en la entrega anterior.

\section{Selenium}
Para realizar el testeo de los casos de uso utilizamos una herramienta que nos permitiera
automatizar la carga de datos y simular la navegación del usuario.
Se adjunta al código fuente una carpeta con todos los tests realizados, separados según
el caso de uso en cuestión.

\end{document}
